% !TeX root = index.tex
% define document class
\documentclass{scrreprt}

% apply pk package
\usepackage{../lib/pk}

% set variables
\newcommand{\varAuthor}{Robin Bühler, Peter Rutschmann}
\newcommand{\varCompany}{} % Firmenname
\newcommand{\varTitle}{Expertencheckliste: für die Durchführung einer IPA als Hauptexperte}
\newcommand{\varVersion}{Version 2025.1}

% footer and headings
\rohead[\varTitle]{\varTitle}
\lofoot[\today]{\today}
\cfoot[\varAuthor\\Version \varVersion]{\varVersion}
\rofoot[Seite \pagemark{} von \pageref{LastPage}]{Seite \pagemark{} von \pageref{LastPage}}

% create document
\begin{document}
\begin{Form}
  \chapter*{\varTitle}
  Bei geschlechterspezifischen Personenbezeichnungen sind immer alle Geschlechter gemeint.\\
  \\
  \sbox\TBox{Arbeitstitel: }\TextField[charsize=10pt, width=\dimexpr\linewidth-\wd\TBox]{Arbeitstitel:}
  \sbox\TBox{Kandidat: }\TextField[charsize=10pt, width=\dimexpr\linewidth-\wd\TBox]{Kandidat:}
  \sbox\TBox{Vr.Fachkraft: }\TextField[charsize=10pt, width=\dimexpr\linewidth-\wd\TBox]{Vr.Fachkraft:}
  \sbox\TBox{Hauptexperte: }\TextField[charsize=10pt, width=\dimexpr\linewidth-\wd\TBox]{Hauptexperte:}
  \sbox\TBox{Nebenexperte: }\TextField[charsize=10pt, width=\dimexpr\linewidth-\wd\TBox]{Nebenexperte:}
  \section{Vorbereitungsphase}
Während der Vorbereitungsphase wird die Aufgabenbeschreibung vom Validierungsexperten betreut. Er ist in Kontakt mit dem Kandidaten, der verantwortlichen Fachkraft und dem Berufsbildner.
\begin{taskitemwithoutcomment}{Auswahl der IPA}
  \textbf{Wähle} die IPA sorgfältig \textbf{aus}.\\Es ist optimal und wird von der verantwortlichen Fachkraft geschätzt, wenn du als Experte einen Bezug zum Thema der Arbeit hast.\\Beachte zudem die \textbf{Ausstands-Regeln}.\\Prüfe, ob die Funktionen des Berufsbildners und der verantwortlichen Fachkraft richtig eingetragen sind.
\end{taskitemwithoutcomment}
\begin{taskitem}{Mithilfe bei der Validierung}
  Beteilige dich an der Validierung. Lese die Aufgabenstellung sorgfältig durch und überlege, ob du den Erfüllungsgrad und die Qualität der Facharbeit beurteilen kannst.\\Der Validierungsexperte ist dankbar für deine Hinweise im PKOrg-Validations-Dialog (nicht in der History).\\Beachte die Netiquette!
\end{taskitem}
\begin{taskitemwithoutcomment}{Deine Verfügbarkeit}
  Bist du während der Durchführung der IPA verfügbar?\\Überprüfe ob die in der Planung der IPA aufgeführten Termine für dich passen und du nicht zBsp. durch Ferien abwesend bist. Das Verschieben der IPA ist keine Option.
\end{taskitemwithoutcomment}
\begin{taskitemwithoutcomment}{Abgabe der IPA}
  Falls du die IPA wieder abgeben musst, dann sind wir froh, wenn du beim Abtausch mithilfst.\\Eventuell kannst du die IPA mit einem anderen Experten abtauschen.
\end{taskitemwithoutcomment}
  \input{sections/2_before-start}
  \newpage
  \section{Erster Besuchstermin durchführen}
Denke daran, dass du bei deinen Besuchen die kantonale Prüfungskommission vertrittst. Korrektes Auftreten, Höflichkeit, Geduld und Pünktlichkeit setzen wir für diese Arbeit voraus. Dein Amt erfordert aber auch Vertraulichkeit und Verschwiegenheit.\\Bereite Dich auf den Besuch vor. Mache während dem Besuch Notizen.

\begin{taskitem}{Vorstellungsrunde}
  Beginne das Gespräch mit einer Vorstellungsrunde. Gebe in erster Linie dem Kandidaten und der verantwortlichen Fachkraft Gelegenheit, sich vorzustellen. Beantworte aber auch Fragen zu deiner Person und zu deinem Arbeitgeber.
\end{taskitem}
\begin{taskitem}{Rollen}
  Stelle die Rollen der an der IPA beteiligten Personen (Kandidat, VF, HEX, NEX, CEX) vor. Erinnere die verantwortliche Fachkraft daran, dass du nun ihr Partner bei der Betreuung des Kandidaten bist.
\end{taskitem}
\begin{taskitemwithoutcomment}{Wichtige Dokumente}
  Haben die verantwortliche Fachkraft und der Kandidat die folgenden IPA-Dokumente gelesen?
  \begin{enumerate}
    \item Dokumente, welche beim Anmelden auf PKOrg bestätigt wurden
    \item QV-Leitfaden von \href{https://pk19.ch}{der Webseite der Prüfungskommission}
    \item QV-Termine von \href{https://pk19.ch}{der Webseite der Prüfungskommission}
  \end{enumerate}
  Diese Dokumente stellen im Kanton Zürich die aktuellen verbindlichen Vorgaben für den Inhalt und die Gestaltung des IPA-Berichts und können sich von anderen Kantonen unterscheiden.
\end{taskitemwithoutcomment}
\newpage
\begin{taskitem}{Arbeitsplatz}
  Achte auf die Infrastruktur und das Umfeld.\\Arbeitet der Kandidat an seinem üblichen Arbeitsplatz? Ist ein ungestörtes Arbeiten möglich?\\Hat der Kandidat neben der IPA keine anderen Aufgaben zu erfüllen?
\end{taskitem}
\begin{taskitem}{Material und Vorarbeiten}
  Sind die Voraussetzungen für die Durchführung der IPA gemäss Aufgabenstellung erfüllt?\\Ist das Material, sind Installationen und Infrastruktur vorhanden und bereit?\\Sind die deklarierten Vorarbeiten erfolgreich abgeschlossen?
\end{taskitem}
\begin{taskitem}{Gemeinsames Verständnis der Aufgabe}
  Hat der Kandidat die Aufgabe verstanden?\\Lasse dir die Aufgabenstellung in Form und Umfang nochmals vom Kandidaten bestätigen.\\Lese auch die Diskussion aus der Validierung nochmals durch. Hat es damals schon Fragen gegeben, die jetzt zu klären sind?
\end{taskitem}
\newpage
\begin{taskitem}{Zeitplan besprechen}
  Kontrolliere, ob der Kandidaten den Zeitplan für die IPA erstellt und als .pdf-Datei auf PKOrg hochgeladen hat.\\Lasse Dir den Zeitplan durch den Kandidaten erklären.
\end{taskitem}
\begin{taskitem}{Bewertungskriterien}
  Bespreche die Bewertungskriterien. Du findest die für diese IPA zusammengestellten Kriterien in der detaillierten Aufgabenstellung.\\Ein gemeinsames Verständnis ist Vorbedingung für eine reibungslose Bewertung.
\end{taskitem}
\begin{taskitem}{Fragen und Probleme}
  Weise darauf hin, dass sich der Kandidat für alle Fragen und Eventualitäten (z.B. Probleme mit Hard- oder Software, aber auch bei Krankheit) an die VF und auch an dich wenden muss. Beachte dazu den Abschnitt weiter hinten in diesem Dokument: \enquote{Zwischenfälle: Pannen, Krankheit, Knapp, ...}).
\end{taskitem}
\newpage
\begin{taskitem}{Nächste Schritte}
  Bespreche die weiteren Termine.\\Der zweite Besuchstermin sollte in der zweiten Hälfte der Facharbeit (Tag 7 oder 8) stattfinden.\\Der dritte Besuchstermin mit der Präsentation, Demonstration, Fachgespräch und Bewertung findet idealerweise etwa eine Woche bis 10 Tage nach Abgabe der IPA-Dokumentation statt.\\Nimm immer auf die Bedürfnisse des Betriebes und des Nebenexperten Rücksicht.\\Findet sich kein geeigneter gemeinsamer Termin mit dem NEX, soll dieser die Arbeit ab- und einem anderen Nebenexperten übergeben.\\Trage alle Termine immer auf PKOrg ein.\\Erkläre wiederum, wie man dich am besten erreichen kann.\\Weise darauf hin, dass die Kommunikation über PKOrg erfolgt.
\end{taskitem}
\begin{taskitemwithoutcomment}{Auftrag an die VF}
  Weise die verantwortliche Fachkraft daran hin, dass der Kandidat sorgfältig bei der Arbeit beobachtet werden muss. Und fordere die verantwortliche Fachkraft auf, ein Protokoll zu führen. Wichtig sind zum Beispiel Notizen zu Hilfestellungen der verantwortlichen Fachkraft während der IPA. Beobachtungen zum Vorgehen, dem Einsatz, den Interaktionen mit dem Team des Kandidaten sowie zu dessen Arbeitszeiten.
\end{taskitemwithoutcomment}

  \newpage
  \section{Zwischenfälle: Pannen, Krankheit, Knappheit, ...}
Zwischenfälle müssen vom Kandidaten oder von der verantwortlichen Fachkraft umgehend dem Hauptexperten gemeldet werden.\\Es ist wichtig, dass alle Zwischenfälle lückenlos über die History und ggf. Anhänge dokumentiert sind.

\begin{enumerate}
  \item Kläre die Umstände möglichst genau ab und finde denkbare Massnahmen.
  \begin{itemize}
    \item Ausfälle durch Verschieben der Abgabe kompensieren.
    \item Anpassen der Aufgabenstellung. Achtung: schriftlich festhalten und ggf. die Bewertungskriterien anpassen.
    \item Abbrechen und später eine neue Facharbeit durchführen
  \end{itemize}
  \item Für Krankheit und Unfall ist ab dem ersten ausfallenden Tag ein Arztzeugnis nötig (Ablage auf PKOrg im Dokumentenpool). Fordere dies sofort ein.
  \item Sprich dich mit dem Chefexperten ab.
  \item Notiere den Entscheid in der PKOrg-History. Trage vereinbarte Termine (Wiederaufnahme der Arbeit, neuer Abgabetermin, ...) in der PKOrg-History ein. Achte darauf, dass alle Beteiligten (Kandidat, VF, EXP) die Information erhalten.
  \item Führe die Besuchstermine auf PkOrg nach, speziell den Präsentationstermin.
  \item Sollte sich zeigen, dass Bewertungskriterien ohne Verschulden des Kandidaten nicht erfüllbar sind, nehme mit dem CEX Kontakt auf.
\end{enumerate}

Falls du das Gefühl hast, dass die Facharbeit ungenügend abgeschlossen werden könnte, solltest Du ...
\begin{itemize}
  \item ... über PKOrg einen Nebenexperten verlangen, falls nicht schon einer zugeteilt ist.
  \item ... deine Bewertung genau begründen, damit die Punktevergabe nachvollziehbar ist. Es muss für Dritte ersichtlich sein, was erreicht wurde oder was fehlt.
  \item ... in deinen Notizen zusätzliche Information festhalten (zBsp. Relevantes für den Fall einer Einsprache oder eines Rekurses). 
  \item ... in besonders heiklen Fällen beim Chefexperten eine Zweitbeurteilung verlangen.

\end{itemize}


Unpünktlichkeit oder gar Fernbleiben macht einen denkbar schlechten Eindruck.\\Deshalb ist es äussert wichtig, alle Beteiligten und bei Bedarf die Prüfungsleitung frühzeitig oder bei Pannen umgehend zu informieren und Ersatzlösungen zu organisieren.

  \newpage
  \section{Zweiter Besuch}
\begin{taskitem}{Vorbereitung für zweiten Besuch}
  Bereite einige Fragen vor. Sie sollen dir unter anderem ein Bild vermitteln, wie weit die Arbeit fortgeschritten ist.
\end{taskitem}
\begin{taskitem}{Situation, Probleme und Hilfestellungen}
  Frag nach, wie der Kandidat den Stand und Situation einschätzt.\\Kann das vorgegebene Ziel erreicht werden?\\Frage nach, welche Problem aufgetaucht und wo allfällige Hilfestellungen notwendig waren?\\Notiere Dir, wie Du den Fortschritt beurteilst.
\end{taskitem}
\begin{taskitem}{Interaktionen}
  Frag nach, welche Interaktionen mit dem Teams stattgefunden haben.\\Wurde der Austausch in Notizen oder Protokollen festgehalten?
\end{taskitem}
\newpage
\begin{taskitem}{Dokumentation, Zeitplan, Arbeitsjournal}
  Schaue dir den Stand der \textbf{Dokumentation} an.\\
  Wie wird der \textbf{Zeitplan} eingehalten? Unterschiede Soll/Ist?\\Wie wird das \textbf{Arbeitsjournal} geführt? Ist es täglich nachgeführt?\\Lass Dir erklären, was noch offen ist.
\end{taskitem}
\begin{taskitem}{Sicherung der Arbeitsartefakte}
  Lass dir zeigen, wie das regelmässige Backup der Dokumentation gelöst wurde.\\Sind die Dokumentversionen identifizierbar?\\Lass dir zeigen, wie eine Version wiederhergestellt werden kann\\Lass dir zeigen, wie andere Artefakte gesichert sind und wie eine Wiederherstellung gemacht werden kann (zBsp GitRepo).
\end{taskitem}
\begin{taskitem}{Stand der Arbeiten, Fachfragen}
  Lasse Dir den Stand der Arbeiten erklären und zeigen.\\Kann schon etwas vorgeführt werden?\\Verwickle den Kandidaten in ein ungezwungenes Fachgespräch.\\Stelle Fragen zum fachlichen Inhalt der Aufgabe. Mache Notizen dazu. Denke dabei auch schon an mögliche Fragestellungen am Präsentationstermin und dem Fachgespräch.
\end{taskitem}
\newpage
\begin{taskitem}{Erwartungen}
  Erkläre dem Kandidaten die Erwartungen an den IPA-Bericht.\\Die Vorgaben im Dokument \enquote{QV-Leitfaden.pdf} sind zu beachten. Gliederung, Umfang, Schwerpunkte und Inhalte.\\Firmennormen/Firmenvorlagen dürfen bzw. sollen angewendet werden.\\Erinnere daran, dass alle Programme und Scripts, die während der Arbeit erstellt wurden, vollständig im Anhang des IPA-Berichts enthalten sein müssen. Das klar ersichtlich sein muss, was der Kandidat davon Rahmen der IPA erstellt hat.
\end{taskitem}
\begin{taskitem}{Vorbereitung dritter Besuchstermin}
  Erkläre die Organisation und den Ablauf der Präsentation, der obligatorischen Demo und des Fachgespräches.\\Frag nach, wo die der Dritte Besuchstermin stattfindet. Kläre ob dies der IPA-Durchführungsadresse auf PKOrg entspricht.\\Frag nach, ob etwas Besonderes notwendig ist, damit alle Beteiligten am dritten Besuch Zugang zu den Räumlichkeiten erhalten.(auch der NEX).
\end{taskitem}
\begin{taskitem}{Abgabetermin}
  Mache den Kandidaten auf den Abgabetermin aufmerksam. Es gilt, was im QV-Leitfaden steht.\\Schau den Abschnitt mit dem Kanditan an.\\Bei Problemen frühzeitig vor dem Abgabetermin bei VF und HEX eskalieren. 
\end{taskitem}
\newpage
\begin{taskitem}{Bewertung der Dokumentation durch die VF}
  Erinnere die verantwortliche Fachkraft daran, dass Sie die Arbeit und den IPA-Bericht vor der Präsentation bewerten muss.\\Die verantwortliche Fachkraft muss im PKOrg ihren Bewertungsvorschlag mit Begründung eintragen und bestätigen. Sie soll zudem in Ihren Notizen Positives und Negatives vermerken.\\Vereinbare einen Termin bis wann die VF das macht. Damit Du danach noch Zeit hast, deine Plausibilisierung zu machen.\\Kläre mit der verantwortlichen Fachkraft allfällige Frage zu den Bewertungskriterien. 
\end{taskitem}
\begin{taskitem}{Themenvorschläge für Fachfragen}
  Bitte die verantwortliche Fachkraft, ca. 3-4 Fachfragen zu unterschiedlichen Themen mit starkem Bezug zur IPA vorzubereiten und auf PKOrg hochzuladen.\\Weise auf das Formular auf PKOrg hin.\\Vereinbare einen Termin, bis wann die VF das macht.\\Weise darauf hin, dass der Kandidat diese Fragen nicht sehen darf.
\end{taskitem}

  \newpage
  \section{Bis zur Präsentation}
Nehme dir Zeit, den IPA-Bericht durchzulesen. Denke daran, das ist ein Bericht einer ausgebildeten Fachkraft.\\Achte beim Lesen auf mögliche Themen für die Fachfragen.

\begin{taskitemwithoutcomment}{Bewertung Bereich 1 und 2}
  Die Teile 1 und 2 wurden bereits von der VF bewertet. Du musst diese Bewertung anhand des abgegebenen Berichts, sowie den Eindrücken bei den Besuchsterminen plausibilisieren und nur Korrekturen zu Bewertung der VF auf POrg eintragen.\\Begründe deine Bewertung nachvollziehbar, insbesondere, wenn Du nicht alle Punkte gibst. Es muss für Dritte klar erkennbar sein, wieso das Puntkemaximum nicht erreicht wurde.
\end{taskitemwithoutcomment}
\begin{taskitemwithoutcomment}{Fachgespräch vorbereiten}
  Bereite Gesprächsthemen für das Fachgespräch vor.\\Nutze das Formular auf PKOrg.\\Berücksichtige auch die Vorschläge der verantwortlichen Fachkraft. So kannst du ein ausgewogenes Fachgespräch vorbereiten.\\Notiere Beispiel-Antworten. Was erwartest du?\\Das Fachgespräch muss einen starken Bezug zur IPA haben. Es ist keine allgemeine Berufskundeprüfung. Zwar ist auch Wissen gefragt, aber immer im Zusammenhang mit der IPA. Das Fachgespräch soll aufzeigen, ob der Kandidat in seiner Fachrichtung und zu seiner IPA kompetent Auskunft geben kann.
\end{taskitemwithoutcomment}
\begin{taskitem}{Fachthemen aneignen}
  Informiere dich falls nötig zusätzlich zu den Themen der IPA. Besonders dann, wenn du mit der verantwortlichen Fachkraft oder dem Kandidaten an den Besuchsterminen zu wenig \enquote{fachsimpeln} konntest.
\end{taskitem}
\begin{taskitemwithoutcomment}{Rolle des Nebenexperten klären}
  Bei der Präsentation, Demonstration, dem Fachgespräch und der Bewertung ist üblicherweise ein zweiter Experte als Nebenexperte dabei. Er sorgt für vollständige Notizen und beteiligt sich am Fachgespräch und der Bewertung.\\Es gibt eine Checkliste für den Nebenexperten auf PKOrg.\\Falls kein Nebenexperte vorhanden ist, kannst du den dritten Besuch auch zu zweit mit der verantwortlichen Fachkraft durchführen.\\Erwartest Du Schwierigkeiten beim dritten Besuch, dann wende die an den Chefexperten, für die Suche nach einem Nebenexperten.
\end{taskitemwithoutcomment}

  \newpage
  \section{Präsentation, Demonstration und Fachgespräch}
Beachte, dass dieser Tag gerade für den Kandidaten ein spezieller Tag ist. Verleihe dem Anlass ruhig etwas Würdevolles.\\Sei pünktlich.\\An der Präsentation, der Demonstration und dem Fachgespräch nehmen der Kandidat, die verantwortliche Fachkraft und die Experten
teil.\\Beachte die Regelungen bezüglich weiteren Personen und der Dauer im QV-Leitfaden.\\Der Hauptexperte leitet durch diesen Anlass.
\begin{taskitem}{Begrüssung}
  Du eröffnest den dritten Besuchstermin als Teil der offiziellen Abschlussprüfung.\\Stelle die Teilnehmende soweit noch nötig vor (zBsp den Nebenexperten).
\end{taskitem}
\begin{taskitem}{Gesundheit}
   Frage den Kandidaten, ob er gesund sei und in der Lage ist, die Prüfung zu absolvieren.\\ Falls nicht, muss der Kandidat zum Arzt und sich ein ärztliches Zeugnis ausstellen lassen.\\In diesem Fall muss ein neuer Termin vereinbart werden.
\end{taskitem}
\begin{taskitemwithoutcomment}{Kriterienkatalog bereit halten}
  Im Kriterienkatalog findest du im Teil 3 die Leitfragen zu der Präsentation und Demonstration mit den Hinweisen, auf was du achten musst.
\end{taskitemwithoutcomment}
\newpage
\begin{taskitemwithoutcomment}{Präsentation}
  Erkläre kurz, was für die Präsentation gilt:\\
  Beachte die Reglung zur Sprache im QV-Leitfaden.\\
  Es werden keine Fragen während der Präsentation gestellt, auch wenn der Kandidat dir das zugestehen würde. Dazu ist später noch Zeit.\\Übergib dem Kandidaten das Wort für den Start der Präsentation. Und wünsche ihm viel Erfolg.\\Miss die Zeit der Präsentation.
\end{taskitemwithoutcomment}
\begin{taskitem}{Beobachtungen zur Präsentation}
  Beachte die Leitfragen im Teil 3 des Kriterienkatalogs.\\
  Achte auf den Aufbau und den Inhalt der Präsentation, aber auch auf die Dauer der Präsentation, den Einsatz der vorhandenen Mittel und den Vortragsstil. Die Kandidaten haben gelernt, wie Vorträge zu halten sind.\\
  Mache dir Notizen.
\end{taskitem}
\newpage
\begin{taskitem}{Demonstration}
  Leite nach der Präsentation über in die Demonstration.\\
  Während der obligatorischen Demo darfst du spontan Fragen stellen. Diese sollten aber den vorbereiteten Ablauf nicht stören.
\end{taskitem}
\begin{taskitem}{Beobachtungen zur Demonstration}
  Beachte die Leitfragen im Teil 3 des Kriterienkatalogs.\\
  Achte auf den Aufbau und den Inhalt der Demonstration.\\
  Mache dir Notizen.
\end{taskitem}
\begin{taskitemwithoutcomment}{Übergang zum Fachgespräch}
  Nach dem Abschluss der Demonstration übernimmst du die Führung für das Fachgespräch.\\Eine kurze Pause wäre vielleicht gut. Das gibt Dir Zeit, die Dokumente zu den Fachfragen bereitzumachen und mit dem Nebenexperten nochmals zu klären, dass er Notizen machen wird.
\end{taskitemwithoutcomment}
\begin{taskitemwithoutcomment}{Durchführung des Fachgesprächs}
  Du als Hauptexperte führst das Gespräch. Achte auf die Zeit gemäss QV-Leitfaden.\\Achte darauf, dass nur der Kandidat die Fragen beantwortet und nicht die VF.\\Notiere (oder besser der Nebenexperte) zu allen Fragen in Stichworten die Antworten bzw. deren Qualität und was gefehlt hat.\\Es müssen 6 bewertbare Fachfragen sein.
\end{taskitemwithoutcomment}
\begin{taskitemwithoutcomment}{Präsentation hochladen, Kandidat verabschieden}
  Bevor du dich vom Kandidaten verabschiedest, bitte ihn, die Präsentation auf PKOrg hochzuladen.\\Erkläre ihm, dass er bei der Bewertung nicht dabei sein darf. Und dass er heute keine Note erfahren wird. Er muss bis Anfangs Juli warten. Denn die Noten werden noch überprüft.\\Bedanke und verabschiede Dich bei ihm.
\end{taskitemwithoutcomment}

  \newpage
  \section{Bewertung}
Die Bewertung muss unmittelbar im Anschluss an das Fachgespräch stattfinden.\\An der Sitzung nehmen nur noch die verantwortliche Fachkraft und die Experten teil.\\In der Regel wird die Bewertung eine gute Stunde dauern.\\Der Hauptexperte leitet das Gespräch.\\Das Ziel ist es, eine Note zu finden, mit welcher die Experten und die verantwortliche Fachkraft einverstanden sind.

\begin{taskitem}{Gesamteindruck}
  Frage die verantwortliche Fachkraft zuerst nach dem Gesamteindruck.
\end{taskitem}
\begin{taskitem}{Teil 3: Präsentation, Demonstration und Fachgespräch auswerten}
  Es macht Sinn, mit der Bewertung der Präsentation, Demonstration und Fachgesprächs zu beginnen.\\Die effektiv geführten Dialoge des Fachgesprächs sind zu gruppieren und zu den vorgesehenen Fachgesprächskriterien zuzuordnen.\\Spontane Fragen bedingen unter Umständen eine Anpassung der vorbereiteten Kriterien. Ergänze die Notizen direkt im Formular zu den Fachfragen.\\Es ist wichtig, dass Dritte die Bewertung der Fachfragen nachvollziehen können.
\end{taskitem}
\newpage
\begin{taskitem}{Bewertung Teil 1 und 2 vergleichen und konsolidieren}
  Anschliessend besprecht ihr die Bewertung von Teil 1 und 2.\\Fokussiere dich auf unterschiedliche Bewertungen zwischen verantwortlicher Fachkraft und Hauptexperte.\\Vermeide Grundsatzdiskussionen und ausschweifendes Fachsimpeln.\\Frage den Nebenexperten nach seiner Meinung.
\end{taskitem}
\begin{taskitemwithoutcomment}{Bewertungsraster auf PKOrg ergänzen}
  Beachte, dass für jedes Kriterium eine nachvollziehbare Begründung eingetragen werden muss. (Bsp. \enquote{unvollständig}: was fehlt oder ist nicht korrekt?)\\Ergänze wo nötig die Kommentare der VF.\\Denke daran: die Punkte repräsentieren keine Note, sondern eine Gütestufe des Kriteriums. Erst die Summe aller Punkte ergibt einen Notenvorschlag.
\end{taskitemwithoutcomment}
\begin{taskitemwithoutcomment}{Einigkeit}
  Der Notenvorschlag erscheint im PKOrg so bald alle Gütestufen und Bemerkungen eingegeben sind.\\Tauscht Euch aus, ob diese Note euren Erwartungen entspricht.\\HEX, NEX und VF müssen signieren. Die VF kann beim Signieren auch vermerken, dass sie mit der Bewertung nicht einverstanden ist.\\Bei Uneinigkeit entscheidet die Notenkonferenz.
\end{taskitemwithoutcomment}
\begin{taskitem}{Rechtzeitige Abgabe des Berichts}
  Stelle fest, ob der Bericht rechtzeitig hochgeladen wurde. Ansonsten gibt es einen Notenabzug nach Vorgaben der Prüfungsleitung.\\Du musst in diesem Fall zwingend den CEX informieren!
\end{taskitem}
\newpage
\begin{taskitemwithoutcomment}{Unterlagen und Notizen auf PKOrg hochladen}
  Lade alle deine Notizen in den Dokumentenpool des Kandidaten auf PKOrg hoch. Achte darauf, dass die verantwortliche Fachkraft und der Nebenexperte dies auch gleich erledigen.
\end{taskitemwithoutcomment}
\begin{taskitemwithoutcomment}{Notenvorschlag vertraulich behandeln}
  Mache die verantwortliche Fachkraft darauf aufmerksam, dass er den Notenvorschlag dem Kandidaten nicht mitteilen darf, weil dieser auch im Nachhinein (beim Quervergleich) durch die Prüfungsleitung verändert werden kann.
\end{taskitemwithoutcomment}
\begin{taskitemwithoutcomment}{Vertraulichkeit}
  Mache die verantwortliche Fachkraft darauf aufmerksam, dass alle Unterlagen vertraulich behandelt werden und im Kreis der Experten der Prüfungskommission verbleiben.
\end{taskitemwithoutcomment}
\begin{taskitemwithoutcomment}{Danksagung}
  Bedanke dich bei der verantwortlichen Fachkraft für ihren Einsatz und ermuntere sie, ebenfalls als Experte mitzuwirken. Weitere Informationen dazu können auf \href{https://pk19.ch}{der Webseite der Prüfungsorganisation} gefunden werden.
\end{taskitemwithoutcomment}
\begin{taskitem}{Austausch mit Nebenexperte}
  Nach Verabschiedung der verantwortlichen Fachkraft nutze die Gelegenheit und besprechen den Ablauf des Prüfungstages kritisch. Beispielfragen: \enquote{Wie war das Verhältnis zum Kandidaten? Wie war der Führungsstil? Gab es Situationen, in der der Kandidat sich unwohl fühlte? War das Fachgespräch flüssig und angemessen? Wo hätte man mehr Aufmerksamkeit schenken können? Strenger/weniger streng reagieren?}
\end{taskitem}

  \newpage
  \section{Abschluss}
Die Prüfungsleitung ist dir dankbar, wenn du die folgenden Arbeiten umgehend erledigst.

\begin{taskitemwithoutcomment}{Verspätete Abgabe}
  Bei verspäteter Abgabe informiere zwingend den Chefexperten per Email.
\end{taskitemwithoutcomment}\begin{taskitemwithoutcomment}{Bemerkungen zum Lehrbetrieb}
  Hast Du Bemerkungen zum Lehrbetrieb. So mache eine Notiz und informieren den CEX.
\end{taskitemwithoutcomment}
\begin{taskitemwithoutcomment}{Uneinigkeit}
  Falls ihr euch bei der Bewertung nicht einig geworden sind. Halte deine Notizen dazu fest und lade sie auf PKOrg hoch.
\end{taskitemwithoutcomment}
\begin{taskitemwithoutcomment}{Digitalisierung}
  Scanne alle deine nicht digitalen IPA-Unterlagen ein und laden diese unter PKOrg bei der IPA im Dokumentenpool hoch, sodass diese für die Prüfungsleitung ersichtlich sind.
\end{taskitemwithoutcomment}
\begin{taskitemwithoutcomment}{Rückfragen}
  Am Tag der Notenklausur solltest du für Rückfragen erreichbar sein. Trage den Termin in deine Agenda ein.
\end{taskitemwithoutcomment}
\begin{taskitemwithoutcomment}{Schweigepflicht}
  Denke an deine Schweigepflicht: Ausserhalb des Bewertungsverfahrens darf weder über den Inhalt der IPA noch über die Bewertung gesprochen werden. Die Rechte am Resultat der IPA gehören der Lehr-/Praktikumsfirma.
\end{taskitemwithoutcomment}
\begin{taskitemwithoutcomment}{Abrechnung}
  Sind alle Arbeiten zu IPA auf PKOrg abgeschlossen. Die notwendigen Signaturen gemacht. Deine Dokumente hochgeladen. Dann Rechne Deine Aufwände auf PKOrg ab.
\end{taskitemwithoutcomment}

\vspace*{2cm}

\textbf{Abschliessend gebührt Dir ein ganz herzliches Dankeschön!}\\Wir sind uns bewusst, dass es nicht selbstverständlich ist, dass du die Arbeit des Prüfungsexperten übernimmst. Die Berufslehre geniesst nicht zuletzt wegen des Prüfungswesens ein hohes Ansehen - schön, dass du diese Verantwortung mitträgst.

\end{Form}
\end{document}
