\section{Abschluss}
Die Prüfungsleitung ist dir dankbar, wenn du die folgenden Arbeiten umgehend erledigst.

\begin{taskitemwithoutcomment}{Verspätete Abgabe}
  Bei verspäteter Abgabe informiere zwingend den Chefexperten per Email.
\end{taskitemwithoutcomment}\begin{taskitemwithoutcomment}{Bemerkungen zum Lehrbetrieb}
  Hast Du Bemerkungen zum Lehrbetrieb. So mache eine Notiz und informieren den CEX.
\end{taskitemwithoutcomment}
\begin{taskitemwithoutcomment}{Uneinigkeit}
  Falls ihr euch bei der Bewertung nicht einig geworden sind. Halte deine Notizen dazu fest und lade sie auf PKOrg hoch.
\end{taskitemwithoutcomment}
\begin{taskitemwithoutcomment}{Digitalisierung}
  Scanne alle deine nicht digitalen IPA-Unterlagen ein und laden diese unter PKOrg bei der IPA im Dokumentenpool hoch, sodass diese für die Prüfungsleitung ersichtlich sind.
\end{taskitemwithoutcomment}
\begin{taskitemwithoutcomment}{Rückfragen}
  Am Tag der Notenklausur solltest du für Rückfragen erreichbar sein. Trage den Termin in deine Agenda ein.
\end{taskitemwithoutcomment}
\begin{taskitemwithoutcomment}{Schweigepflicht}
  Denke an deine Schweigepflicht: Ausserhalb des Bewertungsverfahrens darf weder über den Inhalt der IPA noch über die Bewertung gesprochen werden. Die Rechte am Resultat der IPA gehören der Lehr-/Praktikumsfirma.
\end{taskitemwithoutcomment}
\begin{taskitemwithoutcomment}{Abrechnung}
  Sind alle Arbeiten zu IPA auf PKOrg abgeschlossen. Die notwendigen Signaturen gemacht. Deine Dokumente hochgeladen. Dann Rechne Deine Aufwände auf PKOrg ab.
\end{taskitemwithoutcomment}

\vspace*{2cm}

\textbf{Abschliessend gebührt Dir ein ganz herzliches Dankeschön!}\\Wir sind uns bewusst, dass es nicht selbstverständlich ist, dass du die Arbeit des Prüfungsexperten übernimmst. Die Berufslehre geniesst nicht zuletzt wegen des Prüfungswesens ein hohes Ansehen - schön, dass du diese Verantwortung mitträgst.
