\section{Zwischenfälle: Pannen, Krankheit, Knappheit, ...}
Zwischenfälle müssen vom Kandidaten oder von der verantwortlichen Fachkraft umgehend dem Hauptexperten gemeldet werden.\\Es ist wichtig, dass alle Zwischenfälle lückenlos über die History und ggf. Anhänge dokumentiert sind.

\begin{enumerate}
  \item Kläre die Umstände möglichst genau ab und finde denkbare Massnahmen.
  \begin{itemize}
    \item Ausfälle durch Verschieben der Abgabe kompensieren.
    \item Anpassen der Aufgabenstellung. Achtung: schriftlich festhalten und ggf. die Bewertungskriterien anpassen.
    \item Abbrechen und später eine neue Facharbeit durchführen
  \end{itemize}
  \item Für Krankheit und Unfall ist ab dem ersten ausfallenden Tag ein Arztzeugnis nötig (Ablage auf PKOrg im Dokumentenpool).
  \item Sprich dich mit dem Chefexperten ab.
  \item Notiere den Entscheid in der PKOrg-History. Trage vereinbarte Termine (Wiederaufnahme der Arbeit, neuer Abgabetermin, ...) ein. Achte darauf, dass alle Beteilgigten (Kandidat, VF, EXP) die Information erhalten.
  \item Führe die Besuchstermine auf PkOrg nach, speziell den Präsentationstermin.
  \item Bitte den CEX Prüfungsleitung ggf. die Durchführungstermine auf PKOrg
  anzupassen.
  \item Sollte sich zeigen, dass Bewertungskriterien ohne Verschulden des Kandidaten nicht erfüllbar sind, nehme mit dem CEX Kontakt auf.
\end{enumerate}

Falls du das Gefühl hast, dass die Facharbeit ungenügend abgeschlossen werden könnte, solltest Du ...
\begin{itemize}
  \item ... über PKOrg einen Nebenexperten verlangen, falls nicht schon einer zugeteilt ist.
  \item ... deine Bewertung genau begründen, damit die Punktevergabe nachvollziehbar ist. Es muss für Dritte ersichtlich sein, was erreicht wurde oder was fehlt.
  \item ... in deinen Notizen zusätzliche Infomration festhalten (zBsp Relevantes für den Fall einer Einsprache oder eines Rekurses). 
  \item ... in besonders heiklen Fällen beim Chefexperten eine Zweitbeurteilung verlangen.

\end{itemize}

Unpünktlichkeit oder gar Fernbleiben macht einen denkbar schlechten Eindruck. Deshalb ist es äussert wichtig, alle Beteiligten und bei Bedarf die Prüfungsleitung frühzeitig oder bei Pannen umgehend zu informieren und Ersatzlösungen zu organisieren.
