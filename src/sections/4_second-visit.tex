\section{Zweiter Besuch}
\begin{taskitem}{Vorbereitung für zweiten Besuch}
  Bereite einige Fragen vor. Sie sollen dir unter anderem ein Bild vermitteln, wie weit die Arbeit fortgeschritten ist.
\end{taskitem}
\begin{taskitem}{Situation, Probleme und Hilfestellungen}
  Frag nach, wie der Kandidat den Stand und Situation einschätzt.\\Kann das vorgegebene Ziel erreicht werden?\\Frage nach, welche Problem aufgetaucht und wo allfällige Hilfestellungen notwendig waren?\\Notiere Dir, wie Du den Fortschritt beurteilst.
\end{taskitem}
\begin{taskitem}{Interaktionen}
  Frag nach, welche Interaktionen mit dem Teams stattgefunden haben.\\Wurde der Austausch in Notizen oder Protokollen festgehalten?
\end{taskitem}
\newpage
\begin{taskitem}{Dokumentation, Zeitplan, Arbeitsjournal}
  Schaue dir den Stand der \textbf{Dokumentation} an.\\
  Wie wird der \textbf{Zeitplan} eingehalten? Unterschiede Soll/Ist?\\Wie wird das \textbf{Arbeitsjournal} geführt? Ist es täglich nachgeführt?\\Lass Dir erklären, was noch offen ist.
\end{taskitem}
\begin{taskitem}{Sicherung der Arbeitsartefakte}
  Lass dir zeigen, wie das regelmässige Backup der Dokumentation gelöst wurde.\\Sind die Dokumentversionen identifizierbar?\\Lass dir zeigen, wie eine Version wiederhergestellt werden kann\\Lass dir zeigen, wie andere Artefakte gesichert sind und wie eine Wiederherstellung gemacht werden kann (zBsp GitRepo).
\end{taskitem}
\begin{taskitem}{Stand der Arbeiten, Fachfragen}
  Lasse Dir den Stand der Arbeiten erklären und zeigen.\\Kann schon etwas vorgeführt werden?\\Verwickle den Kandidaten in ein ungezwungenes Fachgespräch.\\Stelle Fragen zum fachlichen Inhalt der Aufgabe. Mache Notizen dazu. Denke dabei auch schon an mögliche Fragestellungen am Präsentationstermin und dem Fachgespräch.
\end{taskitem}
\newpage
\begin{taskitem}{Erwartungen}
  Erkläre dem Kandidaten die Erwartungen an den IPA-Bericht.\\Die Vorgaben im Dokument \enquote{QV-Leitfaden.pdf} sind zu beachten. Gliederung, Umfang, Schwerpunkte und Inhalte.\\Firmennormen/Firmenvorlagen dürfen bzw. sollen angewendet werden.\\Erinnere daran, dass alle Programme und Scripts, die während der Arbeit erstellt wurden, vollständig im Anhang des IPA-Berichts enthalten sein müssen. Das klar ersichtlich sein muss, was der Kandidat davon Rahmen der IPA erstellt hat.
\end{taskitem}
\begin{taskitem}{Vorbereitung dritter Besuchstermin}
  Erkläre die Organisation und den Ablauf der Präsentation, der obligatorischen Demo und des Fachgespräches.\\Frag nach, wo die der Dritte Besuchstermin stattfindet. Kläre ob dies der IPA-Durchführungsadresse auf PKOrg entspricht.\\Frag nach, ob etwas Besonderes notwendig ist, damit alle Beteiligten am dritten Besuch Zugang zu den Räumlichkeiten erhalten.(auch der NEX).
\end{taskitem}
\begin{taskitem}{Abgabetermin}
  Mache den Kandidaten auf den Abgabetermin aufmerksam. Es gilt, was im QV-Leitfaden steht.\\Schau den Abschnitt mit dem Kanditan an.\\Bei Problemen frühzeitig vor dem Abgabetermin bei VF und HEX eskalieren. 
\end{taskitem}
\newpage
\begin{taskitem}{Bewertung der Dokumentation durch die VF}
  Erinnere die verantwortliche Fachkraft daran, dass Sie die Arbeit und den IPA-Bericht vor der Präsentation bewerten muss.\\Die verantwortliche Fachkraft muss im PKOrg ihren Bewertungsvorschlag mit Begründung eintragen und bestätigen. Sie soll zudem in Ihren Notizen Positives und Negatives vermerken.\\Vereinbare einen Termin bis wann die VF das macht. Damit Du danach noch Zeit hast, deine Plausibilisierung zu machen.\\Kläre mit der verantwortlichen Fachkraft allfällige Frage zu den Bewertungskriterien. 
\end{taskitem}
\begin{taskitem}{Themenvorschläge für Fachfragen}
  Bitte die verantwortliche Fachkraft, ca. 3-4 Fachfragen zu unterschiedlichen Themen mit starkem Bezug zur IPA vorzubereiten und auf PKOrg hochzuladen.\\Weise auf das Formular auf PKOrg hin.\\Vereinbare einen Termin, bis wann die VF das macht.\\Weise darauf hin, dass der Kandidat diese Fragen nicht sehen darf.
\end{taskitem}
