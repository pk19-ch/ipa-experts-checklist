\section{Präsentation, Demonstration und Fachgespräch}
Beachte, dass dieser Tag gerade für den Kandidaten ein spezieller Tag ist. Verleihe dem Anlass ruhig etwas Würdevolles.\\Sei pünktlich.\\An der Präsentation, der Demonstration und dem Fachgespräch nehmen der Kandidat, die verantwortliche Fachkraft und die Experten
teil.\\Beachte die Regelungen bezüglich weiteren Personen und der Dauer im QV-Leitfaden.\\Der Hauptexperte leitet durch diesen Anlass.
\begin{taskitem}{Begrüssung}
  Du eröffnest den dritten Besuchstermin als Teil der offiziellen Abschlussprüfung.\\Stelle die Teilnehmende soweit noch nötig vor (zBsp den Nebenexperten).
\end{taskitem}
\begin{taskitem}{Gesundheit}
   Frage den Kandidaten, ob er gesund sei und in der Lage ist, die Prüfung zu absolvieren.\\ Falls nicht, muss der Kandidat zum Arzt und sich ein ärztliches Zeugnis ausstellen lassen.\\In diesem Fall muss ein neuer Termin vereinbart werden.
\end{taskitem}
\begin{taskitemwithoutcomment}{Kriterienkatalog bereit halten}
  Im Kriterienkatalog findest du im Teil 3 die Leitfragen zu der Präsentation und Demonstration mit den Hinweisen, auf was du achten musst.
\end{taskitemwithoutcomment}
\newpage
\begin{taskitemwithoutcomment}{Präsentation}
  Erkläre kurz, was für die Präsentation gilt:\\
  Beachte die Reglung zur Sprache im QV-Leitfaden.\\
  Es werden keine Fragen während der Präsentation gestellt, auch wenn der Kandidat dir das zugestehen würde. Dazu ist später noch Zeit.\\Übergib dem Kandidaten das Wort für den Start der Präsentation. Und wünsche ihm viel Erfolg.\\Miss die Zeit der Präsentation.
\end{taskitemwithoutcomment}
\begin{taskitem}{Beobachtungen zur Präsentation}
  Beachte die Leitfragen im Teil 3 des Kriterienkatalogs.\\
  Achte auf den Aufbau und den Inhalt der Präsentation, aber auch auf die Dauer der Präsentation, den Einsatz der vorhandenen Mittel und den Vortragsstil. Die Kandidaten haben gelernt, wie Vorträge zu halten sind.\\
  Mache dir Notizen.
\end{taskitem}
\newpage
\begin{taskitem}{Demonstration}
  Leite nach der Präsentation über in die Demonstration.\\
  Während der obligatorischen Demo darfst du spontan Fragen stellen. Diese sollten aber den vorbereiteten Ablauf nicht stören.
\end{taskitem}
\begin{taskitem}{Beobachtungen zur Demonstration}
  Beachte die Leitfragen im Teil 3 des Kriterienkatalogs.\\
  Achte auf den Aufbau und den Inhalt der Demonstration.\\
  Mache dir Notizen.
\end{taskitem}
\begin{taskitemwithoutcomment}{Übergang zum Fachgespräch}
  Nach dem Abschluss der Demonstration übernimmst du die Führung für das Fachgespräch.\\Eine kurze Pause wäre vielleicht gut. Das gibt Dir Zeit, die Dokumente zu den Fachfragen bereitzumachen und mit dem Nebenexperten nochmals zu klären, dass er Notizen machen wird.
\end{taskitemwithoutcomment}
\begin{taskitemwithoutcomment}{Durchführung des Fachgesprächs}
  Du als Hauptexperte führst das Gespräch. Achte auf die Zeit gemäss QV-Leitfaden.\\Achte darauf, dass nur der Kandidat die Fragen beantwortet und nicht die VF.\\Notiere (oder besser der Nebenexperte) zu allen Fragen in Stichworten die Antworten bzw. deren Qualität und was gefehlt hat.\\Es müssen 6 bewertbare Fachfragen sein.
\end{taskitemwithoutcomment}
\begin{taskitemwithoutcomment}{Präsentation hochladen, Kandidat verabschieden}
  Bevor du dich vom Kandidaten verabschiedest, bitte ihn, die Präsentation auf PKOrg hochzuladen.\\Erkläre ihm, dass er bei der Bewertung nicht dabei sein darf. Und dass er heute keine Note erfahren wird. Er muss bis Anfangs Juli warten. Denn die Noten werden noch überprüft.\\Bedanke und verabschiede Dich bei ihm.
\end{taskitemwithoutcomment}
